% Options for packages loaded elsewhere
% Options for packages loaded elsewhere
\PassOptionsToPackage{unicode}{hyperref}
\PassOptionsToPackage{hyphens}{url}
%
\documentclass[
  oneside,
  open=any]{scrbook}
\usepackage{xcolor}
\usepackage{amsmath,amssymb}
\setcounter{secnumdepth}{5}
\usepackage{iftex}
\ifPDFTeX
  \usepackage[T1]{fontenc}
  \usepackage[utf8]{inputenc}
  \usepackage{textcomp} % provide euro and other symbols
\else % if luatex or xetex
  \usepackage{unicode-math} % this also loads fontspec
  \defaultfontfeatures{Scale=MatchLowercase}
  \defaultfontfeatures[\rmfamily]{Ligatures=TeX,Scale=1}
\fi
\usepackage{lmodern}
\ifPDFTeX\else
  % xetex/luatex font selection
\fi
% Use upquote if available, for straight quotes in verbatim environments
\IfFileExists{upquote.sty}{\usepackage{upquote}}{}
\IfFileExists{microtype.sty}{% use microtype if available
  \usepackage[]{microtype}
  \UseMicrotypeSet[protrusion]{basicmath} % disable protrusion for tt fonts
}{}
\makeatletter
\@ifundefined{KOMAClassName}{% if non-KOMA class
  \IfFileExists{parskip.sty}{%
    \usepackage{parskip}
  }{% else
    \setlength{\parindent}{0pt}
    \setlength{\parskip}{6pt plus 2pt minus 1pt}}
}{% if KOMA class
  \KOMAoptions{parskip=half}}
\makeatother
% Make \paragraph and \subparagraph free-standing
\makeatletter
\ifx\paragraph\undefined\else
  \let\oldparagraph\paragraph
  \renewcommand{\paragraph}{
    \@ifstar
      \xxxParagraphStar
      \xxxParagraphNoStar
  }
  \newcommand{\xxxParagraphStar}[1]{\oldparagraph*{#1}\mbox{}}
  \newcommand{\xxxParagraphNoStar}[1]{\oldparagraph{#1}\mbox{}}
\fi
\ifx\subparagraph\undefined\else
  \let\oldsubparagraph\subparagraph
  \renewcommand{\subparagraph}{
    \@ifstar
      \xxxSubParagraphStar
      \xxxSubParagraphNoStar
  }
  \newcommand{\xxxSubParagraphStar}[1]{\oldsubparagraph*{#1}\mbox{}}
  \newcommand{\xxxSubParagraphNoStar}[1]{\oldsubparagraph{#1}\mbox{}}
\fi
\makeatother


\providecommand{\tightlist}{%
  \setlength{\itemsep}{0pt}\setlength{\parskip}{0pt}}\usepackage{longtable,booktabs,array}
\usepackage{calc} % for calculating minipage widths
% Correct order of tables after \paragraph or \subparagraph
\usepackage{etoolbox}
\makeatletter
\patchcmd\longtable{\par}{\if@noskipsec\mbox{}\fi\par}{}{}
\makeatother
% Allow footnotes in longtable head/foot
\IfFileExists{footnotehyper.sty}{\usepackage{footnotehyper}}{\usepackage{footnote}}
\makesavenoteenv{longtable}
\usepackage{graphicx}
\makeatletter
\newsavebox\pandoc@box
\newcommand*\pandocbounded[1]{% scales image to fit in text height/width
  \sbox\pandoc@box{#1}%
  \Gscale@div\@tempa{\textheight}{\dimexpr\ht\pandoc@box+\dp\pandoc@box\relax}%
  \Gscale@div\@tempb{\linewidth}{\wd\pandoc@box}%
  \ifdim\@tempb\p@<\@tempa\p@\let\@tempa\@tempb\fi% select the smaller of both
  \ifdim\@tempa\p@<\p@\scalebox{\@tempa}{\usebox\pandoc@box}%
  \else\usebox{\pandoc@box}%
  \fi%
}
% Set default figure placement to htbp
\def\fps@figure{htbp}
\makeatother

\makeatletter
\@ifpackageloaded{caption}{}{\usepackage{caption}}
\AtBeginDocument{%
\ifdefined\contentsname
  \renewcommand*\contentsname{Table of contents}
\else
  \newcommand\contentsname{Table of contents}
\fi
\ifdefined\listfigurename
  \renewcommand*\listfigurename{List of Figures}
\else
  \newcommand\listfigurename{List of Figures}
\fi
\ifdefined\listtablename
  \renewcommand*\listtablename{List of Tables}
\else
  \newcommand\listtablename{List of Tables}
\fi
\ifdefined\figurename
  \renewcommand*\figurename{Figure}
\else
  \newcommand\figurename{Figure}
\fi
\ifdefined\tablename
  \renewcommand*\tablename{Table}
\else
  \newcommand\tablename{Table}
\fi
}
\@ifpackageloaded{float}{}{\usepackage{float}}
\floatstyle{ruled}
\@ifundefined{c@chapter}{\newfloat{codelisting}{h}{lop}}{\newfloat{codelisting}{h}{lop}[chapter]}
\floatname{codelisting}{Listing}
\newcommand*\listoflistings{\listof{codelisting}{List of Listings}}
\makeatother
\makeatletter
\makeatother
\makeatletter
\@ifpackageloaded{caption}{}{\usepackage{caption}}
\@ifpackageloaded{subcaption}{}{\usepackage{subcaption}}
\makeatother

\usepackage{hyphenat}
\usepackage{ifthen}
\usepackage{calc}
\usepackage{calculator}

\usepackage{graphicx}
\usepackage{wallpaper}

\usepackage{geometry}

\usepackage{graphicx}
\usepackage{geometry}
\usepackage{afterpage}
\usepackage{tikz}
\usetikzlibrary{calc}
\usetikzlibrary{fadings}
\usepackage[pagecolor=none]{pagecolor}


% Set the titlepage font families







% Set the coverpage font families

\usepackage{bookmark}
\IfFileExists{xurl.sty}{\usepackage{xurl}}{} % add URL line breaks if available
\urlstyle{same}
\hypersetup{
  pdftitle={A Sample Title - The SocioEconomic Aspects of Stock Assessments},
  pdfauthor={Johan VB; Eva Nováková; Matti Meikäläinen; Ashok Kumar},
  hidelinks,
  pdfcreator={LaTeX via pandoc}}


\title{A Sample Title - The SocioEconomic Aspects of Stock Assessments}
\usepackage{etoolbox}
\makeatletter
\providecommand{\subtitle}[1]{% add subtitle to \maketitle
  \apptocmd{\@title}{\par {\large #1 \par}}{}{}
}
\makeatother
\subtitle{with non-English diacritics in the author names. See
documentation.}
\author{Johan VB \and Eva Nováková \and Matti Meikäläinen \and Ashok
Kumar}
\date{}
\begin{document}
%%%%% begin titlepage extension code

  \begin{frontmatter}

\begin{titlepage}

%%% TITLE PAGE START

% Set up alignment commands
%Page
\newcommand{\titlepagepagealign}{
\ifthenelse{\equal{left}{right}}{\raggedleft}{}
\ifthenelse{\equal{left}{center}}{\centering}{}
\ifthenelse{\equal{left}{left}}{\raggedright}{}
}


\newcommand{\titleandsubtitle}{
% Title and subtitle
{{\large{\bfseries{\nohyphens{A Sample Title - The SocioEconomic Aspects
of Stock Assessments}}}}\par
}%

\vspace{\betweentitlesubtitle}
{
{\large{\textit{\nohyphens{with non-English diacritics in the author
names. See documentation.}}}}\par
}}
\newcommand{\titlepagetitleblock}{
\titleandsubtitle
}

\newcommand{\authorstyle}[1]{{\large{#1}}}

\newcommand{\affiliationstyle}[1]{{\large{#1}}}

\newcommand{\titlepageauthorblock}{
{\authorstyle{\nohyphens{Johan
VB}{\textsuperscript{1}}\textsuperscript{,}{\textsuperscript{2}},  \nohyphens{Eva
Nováková}{\textsuperscript{3}},  \nohyphens{Matti
Meikäläinen}{\textsuperscript{4}}\textsuperscript{,}{\textsuperscript{*}} and \nohyphens{Ashok
Kumar}{\textsuperscript{2}}\textsuperscript{,}{\textsuperscript{5}}}}}

\newcommand{\titlepageaffiliationblock}{
\hangindent=1em
\hangafter=1
{\affiliationstyle{
{1}.~Minnesota Department of Natural Resources,~500 Lafayette Road Saint
Paul, MN 55155
\par\hangindent=1em\hangafter=1{2}.~University of Minnesota,~Department
of Mathematics
\par\hangindent=1em\hangafter=1{3}.~Czech University of Life
Sciences,~Družstevní 666, Vikýřovice, Czechia
\par\hangindent=1em\hangafter=1{4}.~University of Kemijärvi,~Department
of Biological and Environmental Science,~Kylmäniementie 79, 98120,
KEMIJÄRVI, Finland
\par\hangindent=1em\hangafter=1{5}.~HØnefoss Institute,~R Tradição 35,
Portugal 2950-726


\vspace{1\baselineskip} 
* \textit{Correspondence:}~Matti Meikäläinen~matti@jy.fi
}}
}
\newcommand{\headerstyled}{%
{The Publisher}
}
\newcommand{\footerstyled}{%
{\large{NOAA Fisheries OpenSci\\
Tools for Open Science\\
\url{https://github.com/nmfs-opensci}\strut \\}}
}
\newcommand{\datestyled}{%
{}
}


\newcommand{\titlepageheaderblock}{\headerstyled}

\newcommand{\titlepagefooterblock}{
\footerstyled
}

\newcommand{\titlepagedateblock}{
\datestyled
}

%set up blocks so user can specify order
\newcommand{\titleblock}{\newlength{\betweentitlesubtitle}
\setlength{\betweentitlesubtitle}{\baselineskip}
{

{\titlepagetitleblock}
}

\vspace{4\baselineskip}
}

\newcommand{\authorblock}{{\titlepageauthorblock}

\vspace{2\baselineskip}
}

\newcommand{\affiliationblock}{{\titlepageaffiliationblock}

\vspace{1pt}
}

\newcommand{\logoblock}{{\includegraphics[width=0.25\textheight]{img/logo.png}}

\vspace{2\baselineskip}
}

\newcommand{\footerblock}{{\titlepagefooterblock}

\vspace{1pt}
}

\newcommand{\dateblock}{}

\newcommand{\headerblock}{{\titlepageheaderblock

\vspace{0pt}
}}
\newgeometry{top=3in,bottom=1in,right=1in,left=1in}
% background image
\newlength{\bgimagesize}
\setlength{\bgimagesize}{0.5\paperwidth}
\LENGTHDIVIDE{\bgimagesize}{\paperwidth}{\theRatio} % from calculator pkg
\ThisULCornerWallPaper{\theRatio}{img/corner-bg.png}

\thispagestyle{empty} % no page numbers on titlepages


\newcommand{\vrulecode}{\textcolor{black}{\rule{\vrulewidth}{\textheight}}}
\newlength{\vrulewidth}
\setlength{\vrulewidth}{1pt}
\newlength{\B}
\setlength{\B}{\ifdim\vrulewidth > 0pt 0.05\textwidth\else 0pt\fi}
\newlength{\minipagewidth}
\ifthenelse{\equal{left}{left} \OR \equal{left}{right} }
{% True case
\setlength{\minipagewidth}{\textwidth - \vrulewidth - \B - 0.1\textwidth}
}{
\setlength{\minipagewidth}{\textwidth - 2\vrulewidth - 2\B - 0.1\textwidth}
}
\ifthenelse{\equal{left}{left} \OR \equal{left}{leftright}}
{% True case
\raggedleft % needed for the minipage to work
\vrulecode
\hspace{\B}
}{%
\raggedright % else it is right only and width is not 0
}
% [position of box][box height][inner position]{width}
% [s] means stretch out vertically; assuming there is a vfill
\begin{minipage}[b][\textheight][s]{\minipagewidth}
\titlepagepagealign
\titleblock

\authorblock

\affiliationblock

\vfill

\logoblock

\footerblock
\par

\end{minipage}\ifthenelse{\equal{left}{right} \OR \equal{left}{leftright} }{
\hspace{\B}
\vrulecode}{}
\clearpage
\restoregeometry
%%% TITLE PAGE END
\end{titlepage}
\setcounter{page}{1}
\end{frontmatter}

%%%%% end titlepage extension code

\renewcommand*\contentsname{Table of contents}
{
\setcounter{tocdepth}{2}
\tableofcontents
}
\listoffigures
\listoftables

\mainmatter
\chapter{1 The Assignment}\label{the-assignment}

\section{1.1 Company Profile}\label{company-profile}

Describe the commissioning organization (Value Chain Hackers Lab within
Windesheim). Include mission, role, and relevant context.

\section{1.2 Problem Statement}\label{problem-statement}

Explain the specific operational or innovation problem. Show why it
matters at micro, meso, and macro level. Support with evidence.

\section{1.3 The Goal and Objective}\label{the-goal-and-objective}

State the overarching goal and concrete objectives. Explain what the
project should achieve and how success can be measured.

\section{1.4 Match to the Micro Innovation Track and Personal
Fit}\label{match-to-the-micro-innovation-track-and-personal-fit}

Explain why this is a proper Micro Innovation assignment. Show how it
connects to your interests, strengths, and future ambitions.

\section{1.5 Confirmed Client
Requirements}\label{confirmed-client-requirements}

Summarize the requirements agreed with the client. Mention scope,
expectations, deliverables, and validation needs.

\section{1.6 Ambition Statement}\label{ambition-statement}

Clarify the ambition of the project in professional terms. Show the
added value for client and your own growth.

\section{1.7 BI Meta-skills Mapping}\label{bi-meta-skills-mapping}

Explain how you will apply and demonstrate Define, Design, Execute,
Learn, Lead. Use examples from the assignment.

\section{1.8 Trends and Developments}\label{trends-and-developments}

Summarize relevant external trends, developments, and scenarios (past,
present, near future) that justify the project.

\section{1.9 Why This Assignment Is
Important}\label{why-this-assignment-is-important}

Conclude why this project matters --- to the lab, partners, and your own
development.

\chapter{2 Research Design}\label{research-design}

\section{2.1 Main Research Question}\label{main-research-question}

Formulate one clear and researchable question.

\section{2.2 Sub-questions}\label{sub-questions}

Provide several sub-questions that together answer the main one. Cover
5W1H logic.

\section{2.3 Hypothesis}\label{hypothesis}

State a testable proposition based on the expected effect of your
solution.

\section{2.4 Research Methodology}\label{research-methodology}

Describe the approach, data sources, research population, collection,
and analysis methods. Show feasibility.

\section{2.5 Validation Logic from Previous
Research}\label{validation-logic-from-previous-research}

Show how prior studies/frameworks will inform your design. Mention
benchmarking and feasibility testing.

\section{2.6 Filing System and Data Management
Approach}\label{filing-system-and-data-management-approach}

Explain how data, notes, and results will be stored, versioned, and
shared.

\chapter{3 Project Design}\label{project-design}

\section{3.1 Project Planning and
Timeline}\label{project-planning-and-timeline}

Present the timeline across the full graduation project (milestones,
sprints, phases).

\section{3.2 Stakeholder Analysis and Role
Division}\label{stakeholder-analysis-and-role-division}

Identify stakeholders, what is at stake for each, and their
roles/responsibilities.

\section{3.3 Required Resources}\label{required-resources}

List resources (tools, access, support) and assess feasibility.

\section{3.4 Risks and Mitigation
Strategies}\label{risks-and-mitigation-strategies}

Identify obvious and less obvious risks, and how you will deal with
them.

\section{3.5 Criteria and Specifications for
Deliverables}\label{criteria-and-specifications-for-deliverables}

Define scope, criteria, and specifications for final outputs and
intermediate results.

\section{3.6 Milestones and
Checkpoints}\label{milestones-and-checkpoints}

Set out key milestones, expected outputs, validation actions, and
timing.

\section{3.7 Success Criteria}\label{success-criteria}

Explain how success will be judged by both client and assessment board.

\section{3.8 Location for Shared Data, Files, and
Updates}\label{location-for-shared-data-files-and-updates}

State where materials will be stored and how progress will be monitored.

\chapter{4 APA Reference List}\label{apa-reference-list}

Provide all references in APA style. Ensure accuracy and consistency.

\chapter{5 Appendix}\label{appendix}

Include supporting material such as transcripts, templates, stakeholder
input, or planning visuals.


\backmatter


\end{document}
